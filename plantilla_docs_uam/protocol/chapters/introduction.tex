\section{Antecedentes}
Describir de manera exhaustiva el contexto histórico y actual del relativo al problema que desea resolverse. Incluir un resumen de investigaciones pasadas y su relevancia, de preferencia hacer un cuadro comparativo de las investigaciones.

Iniciar con un contexto muy general e ir particularizando hasta llegar al contexto inmediato de la problemática. Si la propuesta es parte de un proyecto más grande, al final de esta sección en donde se debe dar toda la descripción de dicho proyecto y hacer hincapié en dónde se inserta la propuesta que se está describiendo.

\section{Propuesta}
\textbf{Planteamiento del problema.}
Definir claramente la problemática que que se pretende resolver. Debe ser sucinto, medible y alcanzable. Explicar por qué es un problema significativo en el campo de la ingeniería y cómo su solución podría contribuir al avance del conocimiento y sus posibles aplicaciones clínicas, en el contexto real o como parte del proyecto general si es que la propuesta es un subproyecto.

\textbf{Justificación.}
Detallar las razones que avalan la necesidad de la ejecución de la propuesta. Algunos factores pueden ser: relevancia académica, potencial de impacto social o industrial. También deben describirse las implicaciones y qué es lo que se podrá realizar una vez que la propuesta se haya concluido.

\textbf{Objetivo General.}
El Objetivo General es un enunciado que expresa de forma clara y sucinta que es lo que se espera alcanzar al desarrollar la propuesta. La formula de redacción puede sintetizarse en: 1)~Verbo en infinitivo que condensa todo lo que involucrará el desarrollo y ejecución de la propuesta. 2)~Objeto de estudio o elemento resolutivo, en este caso se establece qué se estudiará o qué elemento deberá desarrollarse para poder solucionar el problema. 3)~Finalidad de los dos elementos anteriores, es decir, se detalla el para qué de dichos elementos. 4)~Población de estudio, breve descripción de la población que será directamente estudiada. 5)~Descripción de los datos, en el caso que aplique en esta parte se describen las características de los datos de donde se extrajo la población de estudio. A continuación un ejemplo de la aplicación de dicha formula para redactar un objetivo general:

\begin{quote}
Implementar y evaluar una Red Neuronal Profunda de la familia RestNet para la segmentación automatizada de estructuras en fetos de 20 a 34 semanas de gestación a partir de volúmenes de ultrasonido 3D obtenidos en consulta de seguimiento de madres gestantes con diagnóstico de preeclampsia.
\end{quote}

\textbf{Objetivos Específicos}
Para generar los objetivos específicos tome en cuenta la estrategia de programación estructurada \textit{divide y vencerás} la cual consiste en tomar un problema y partirlo en varios problemas más pequeños. Con esto en mente, se sugiere que tome al Objetivo General y segmentarlo en tres o cuatro subobjetivos, en otras palabras, al completar los subobjetivos planteados se estará completando el Objetivo General.

Estos subobjetivos son los denominados Objetivos Específicos los cuales deben ser concretos, sucintos y que permitan medir el avance, es decir, al finalizarlos se tendrá un resultado como son: software, prototipos, modelos, gráficas, cuadros, etc. Considere que los reportes (tesis, tesinas o artículos) y validaciones (de algoritmos, scripts desarrollados, modelos de clasificadores o de regresión) son inherentes al desarrollo de la propuesta por lo que no deben considerarse como objetivos específicos, pero si como tareas a realizar, esto se explicará más adelante. Para redactar un objetivo específico puede usarse una estructura similar a la del Objetivo General, pero posiblemente podrían omitirse los incisos 4 o 5.


\textbf{Hipótesis o Hipótesis de Trabajo}
Es importante distinguir entre dos tipos de propuestas: 1)~Investigación, en estas se plantea una Hipótesis, es decir, un enunciado escrito como una afirmación gramaticalmente correcta a ad hoc al contexto, construido a partir del conocimiento previo del área de estudio, con la salvedad de que la validez de dicha afirmación no ha corroborada por lo que el objetivo de la propuesta es decir si la afirmación es válida (hipótesis nula) o no (hipótesis alternativa). Es por esto último que la hipótesis debe redactarse de tal forma que la afirmación sea dicotómica. 2)~Desarrollo, este tipo de propuestas no está enfocado en la generación de conocimiento nuevo, sino en el desarrollo de herramientas tecnológicas de hardware o software, así como replicar otras investigaciones, por tal motivo se suelen utilizar las Hipótesis de Trabajo, estas a diferencia de su contraparte del ámbito de investigación, son todos los supuestos que se deben cumplir para que pueda la propuesta pueda ser desarrollada cabalmente.

Antes de enunciar la hipótesis es importante mencionar qué tipo de propuesta es la que se está realizando. Tome en cuenta que tanto para una propuesta de ingeniería como de maestría no es absolutamente necesaria la generación de nuevo conocimiento, esto no aplica para una propuesta doctoral.

\section{Metodología Propuesta}
Describir en detalle los métodos y técnicas que se utilizarán para llevar a cabo la investigación. Incluir el diseño del experimento, los métodos de recolección de datos, y los procedimientos para el análisis de los datos. Justificar por qué estos métodos son los más adecuados para alcanzar tus objetivos.

\section{Material Requerido}
Proporcionar una lista detallada de todos los materiales, equipos y software necesarios para la investigación. Especificar modelos, cantidades y cualquier otro detalle relevante para entender completamente los requerimientos de la investigación.

\section{Cronograma de Actividades}
Presentar un cronograma detallado que muestre el plan de trabajo para la investigación. Incluir fechas, duraciones y descripciones de cada actividad o fase del proyecto. Utilizar herramientas como un diagrama de Gantt para una representación visual.

\end{document}
